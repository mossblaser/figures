% `hexagons' TikZ Library
% (c) 2013, Jonathan Heathcote
%
% A TikZ library for drawing hexagonal stuff, principally for the benefit of the
% SpiNNaker group.

% This library is the underlying method of drawing hexagonal nodes
\usetikzlibrary{shapes}

% When hexagon nodes have their 'minimum size' set it referrs to the diameter
% of a circle which touches the points of the hexagon. Since our hexagons will
% be touching at their edges (not the points) this scale factor corrects this
% so that 'minimum size' reffers instead to the circle that fits inside the
% hexagon.
\pgfmathsetmacro{\hexRadiusScale}{1/cos(30)}

% An attribute to be added to a node which causes it to be hexagonal with the
% flat edge on top. The size of the hexagon should be set with 'minimum size' as
% usual.
\tikzset{
	hexagonBoard/.style={ regular polygon
	                    , regular polygon sides=6 % Hexagon!
	                    , scale = \hexRadiusScale % Correct for minimum size
	                    }
}

% An attribute to be added to a node which causes it to be hexagonal. The size
% of the hexagon should be set with 'minimum size' as usual.
\tikzset{
	hexagon/.style={ hexagonBoard
	               , regular polygon rotate=30 % Point-on-top
	               }
}

% This attribute should be added to your environment and it sets up the XYZ
% coordinate system to correspond to the coordinate system conventionally used
% with hexagonBoards.
\tikzset{
	hexagonBoardXYZ/.style={ x=( -30:1cm)
	                       , y=(  90:1cm)
	                       , z=(-150:1cm)
	                       }
}

% This attribute should be added to your environment and it sets up the XYZ
% coordinate system to correspond to the coordinate system conventionally used
% with board hexagons.
\tikzset{
	hexagonXYZ/.style={ x=(   0:1cm)
	                  , y=( 120:1cm)
	                  , z=(-120:1cm)
	                  }
}

% --- Begin 3rd Party Code ---
% Code to add aliases for anchor names
% By Martin Scharrer (http://tex.stackexchange.com/a/14772)

\def\pgfaddtoshape#1#2{%
	\begingroup
	\def\shape@name{#1}%
	\let\anchor\pgf@sh@anchor
	#2%
	\endgroup
}

\newcommand{\anchoralias}[2]{%
	\expandafter
	\gdef\csname pgf@anchor@\shape@name @#1\expandafter\endcsname
	\expandafter{\csname pgf@anchor@\shape@name @#2\endcsname}%
}
% --- End 3rd Party Code ---


% Add aliases for the sides of the hexagon as preferred by the SpiNNaker
% conventions
\pgfaddtoshape{regular polygon}{%
	\anchoralias{side north}{side 1}%
	\anchoralias{side south}{side 4}%
	\anchoralias{side west}{side 2}%
	\anchoralias{side east}{side 5}%
	\anchoralias{side north east}{side 6}%
	\anchoralias{side south west}{side 3}%
}

